%%% Local Variables:
%%% mode: latex
%%% TeX-master: "tutorials.tex"
%%% End:

\begin{question}
  Verify the dimensions in both the $FLT$ and $MLT$ systems, of the
  following quantities which appear in Table 1.1:
  \begin{enumerate}[label=\alph*)]
  \item volume,
  \item acceleration,
  \item mass,
  \item moment of inertia (area),
  \item work
  \end{enumerate}
\end{question}
\begin{solution}
  \begin{enumerate}[label=\alph*)]
  \item Since volume is the result of cubing the dimension of length,
    its dimension will be $\displaystyle [L^3]$
  \item Acceleration is defined as the time rate of change of
    velocity. Thus its dimension will be equal to the dimension of
    velocity divided by the dimension of time.
    \begin{equation*}
      \Rightarrow \frac{[L\,T^{-1}]}{[T]}\,\Rightarrow [L\,T^{-2}]
    \end{equation*}
  \item Mass is one of the fundamental dimensions ($MLT$) and hence its
    dimension is simply $M$. We can convert this to the $FLT$ system
    using Newton's second law: $F = ma$ where $F$ is force and $m$ is
    mass and $a$ is acceleration. Using dimensions, we get:
    \begin{align*}
      [F] &= [M][L\,T^{-2}] \\
      \Rightarrow [M] &= [F\,L^{-1}\,T^2]
    \end{align*}
  \item Area moment of inertia, also called the second moment of
    area is a measure of an object resistance to change. For example
    the second moment of area about the $x-$axis is given by:
    \begin{equation*}
      I_{xx} = \int y^2 \,dy\,dx
    \end{equation*}
    Where y is the distance to the axis. It's clear from the
    equation that the dimension is $\displaystyle [L^4]$
  \item Work is a product of force and distance. Hence, its dimension
    is $[F\,L]$. In the $MLT$ system, the dimension of work is $[M\,L^2\,T^{-2}]$
  \end{enumerate}
\end{solution}

\begin{question}
  If $P$ is a force and $x$ a length, what are the dimensions (in the
  $FLT$ system) of:
  \begin{enumerate}[label=\alph*)]
  \item \dod{P}{x}
  \item \dod[3]{P}{x}
  \item $\int P\,\dif x$
  \end{enumerate}
\end{question}
\begin{solution}
  \begin{enumerate}[label=\alph*)]
  \item This a rate of change of $P$ with respect to $x$. Finding
    the rate of a quantity \emph{does not change} its dimension
    (e.g. the dimension of $P$ is $[L]$ and the dimension of $dp$ is
    also $[L]$). The dimension of \dod{P}{x} is:
    \begin{equation*}
      \frac{[F]}{[L]} \Rightarrow [F\,L^{-1}]
    \end{equation*}
  \item This is the \emph{third derivative} of $P$ with respect to
    $x$. It is easy to make a mistake in determining the dimension of
    this quantity but by simplifying the expression, the dimension
    will be clear:
    \begin{equation*}
      \dod[3]{P}{x} = \dod{}{x}\left(\dod{}{x}\left(\dod{P}{x}\right)\right)
    \end{equation*}
    It can be seen that the dimension of $\dif^3P$ does not change
    because it is the rate of the rate of $\dif P$. For the denominator,
    it's different as we are multiplying rates hence the dimension of
    $\dif x^3$ is $[L^3]$. Finally, we get the dimension to be $[F\,L^{-3}]$.
  \item The integral evaluates to $P\,x$ (plus a constant but we can
    ignore it here). The dimension of the integral is $[F\,L]$.
  \end{enumerate}
\end{solution}


\begin{question}
  Determine the dimension of the coefficients $A$ and $B$ which appear
  in the dimensionally homogeneous equation
  \begin{equation*}
    \dod[2]{x}{t} + A\dod{x}{t} + Bx = 0
  \end{equation*}
  Where $x$ is a length and $t$ is time.
\end{question}
\begin{solution}
  The equation is dimensionally homogeneous which means all terms in
  the equation must have the same dimension. The dimension of
  \dod[2]{x}{t} is $[L\,T^{-2}]$ and all other terms should have
  this dimension:

  Dimension of $A\dod{x}{t}$ is $[L\,T^{-2}]$, knowing the dimension
  of $\dod{x}{t}$ to be $[L\,T^{-1}]$. We get the dimension of $A$:
  $[T^{-1}]$

  Dimension of $Bx$ is also $[L\,T^{-2}]$. $x$ is a length so its
  dimension is $[L]$ and hence the dimension of $B$ is $[T^{-2}]$
\end{solution}

\begin{question}
  According to information found in an old hydraulics book, the energy
  loss per unit weight of fluid flowing through a nozzle connected to
  a hose can be estimated by the formula
  \begin{equation*}
    h = (\num{0.04} \textrm{ to } \num{0.09})(D/d)^4V^2/2g
  \end{equation*}
  where $h$ is the energy loss per unit weight, $D$ the hose diameter,
  $d$ the nozzle tip diameter, $V$ the fluid velocity in the hose, and
  $g$ the acceleration of gravity. Do you think this equation is valid
  in any system of units? Explain.
\end{question}
\begin{solution}
  Energy has the dimension $[F\,L]$ and thus energy per unit weight
  has a dimension $[\frac{F\,L}{F}]$. The dimension of other terms in
  the equation has already been found previously. By substituting each
  dimension in the equation we get:
  \begin{align*}
    \left[\frac{F\,L}{F}\right] &=
    (\num{0.04}\textrm{ to } \num{0.09})\left[\frac{L^4}{L^4}\right](\frac{1}{2})
    \left[\frac{L^2}{T^2}\right]\left[\frac{T^2}{L}\right] \\
    [L] &= [L]
  \end{align*}
 The dimensions are equal and hence this equation is dimensionally
 homogeneous and is valid in any system of units.
\end{solution}

\begin{question}
  Make use of Table 1.3 to express the following quantities in SI
  units:
  \begin{enumerate}[label=\alph*)]
  \item \SI{10.2}{\inch/\minute}
  \item \SI{4.81}{\slug}
  \item \SI{3.02}{\pound}
  \item \SI{73.1}{\feet/\s^2}
  \item \SI{0.0234}{\pound\cdot{}s/\feet^2}
  \end{enumerate}
\end{question}
\begin{solution}
  \begin{enumerate}[label=\alph*)]

  \item
    \begin{align*}
      \US{10.2}{\inch/\minute} &\Rightarrow (\US{10.2}{\inch/\minute})
    (\SI{2.540e-2}{\m/\inch})(\SI[fraction-function=\dfrac]{1/60}{\minute/\s})
      \\
&= \SI{4.32e-3}{\m/\s} = \SI{4.32}{\mm/\s}
    \end{align*}
    \item
      \begin{align*}
        \US{4.81}{\slug} &\Rightarrow
                           (\US{4.81}{\slug})(\SI{14.59}{\kg/\slug})
        \\
        &= \SI{70.2}{\kg}
      \end{align*}
\item
  \begin{align*}
    \US{3.02}{\pound} &\Rightarrow
                        (\US{3.02}{\pound})(\SI{4.448}{\N/\pound}) \\
    &= \SI{13.4}{\N}
  \end{align*}
\item
  \begin{align*}
    \SI{73.1}{\feet/\s^2} &\Rightarrow (\SI{73.1}{\feet/\s^2})
                          (\SI{3.0481e-1}{\myunitA\per\myunitB}) \\
                          &= \SI{22.3}{\m/\s^2}
  \end{align*}
\item
  \begin{align*}
    \SI{0.0234}{\myunitE} &\Rightarrow (\SI{0.0234}{\myunitE})
(\SI{47.88}{\myunitD\per\myunitE}) \\
&= \SI{1.12}{\myunitD}
  \end{align*}
  \end{enumerate}
\end{solution}

\begin{question}
  An important dimensionless parameter in certain types of fluid flow
  problems is the Froude number defined as $V/\sqrt{gl}$, where $V$ is
  a velocity, $g$ the acceleration of gravity, and $l$ a
  length. Determine the value of the Froude number for $V =
  \SI{10}{\feet/\s}$, and $l=\SI{2}{\feet}$. Recalculate the Froude
  number using SI units for $V, g$ and $l$. Explain the significance
  of the results of these calculations.
\end{question}

\begin{question}
  The information on a can of pop indicates that the can contains
  \SI{355}{\mL}. The mass of a full can of pop is \SI{0.369}{\kg}
  while an empty can weighs \SI{0.153}{\N}. Determine the specific
  weight, density, and specific gravity of the pop and compare your
  results with the corresponding values for water at
  \SI{20}{\celsius}. Express your results in SI units.
\end{question}
\begin{solution}
  We first find the specific weight using the equation:
  \begin{equation*}
    \gamma = \frac{\textrm{weight of fluid}}{\textrm{volume of fluid}}
  \end{equation*}
  Weight of the fluid can be obtained by subtracting the weight of the
  can from the total weight (weight of can $+$ weight of pop). We are
  given the weight of the can ($\SI{0.153}{\N}$) but not the total
  weight. A total mass is rather given $(\SI{0.369}{\kg})$. We can
  find the weight using:
  \begin{align*}
    &W_{total} = m_{total} * g \\
    &\textrm{Where $g$ is the acceleration due to gravity.} \\
    &W_{total} =  \SI{3.62}{\N} \\
    &\therefore \textrm{weight of fluid} = \num{3.62} - \num{0.153}\si{\N} =
    \SI{3.467}{\N}
\end{align*}
We now need the volume of fluid, it is given in units of
\si{\mL}. This needs to be converted to \si{\m^3}.
\begin{align*}
 \SI{1}{\m^3} &= \SI{e6}{\mL^3}
\therefore \SI{355}{\mL} &= \SI{355e-6}{\m^3}
\end{align*}
Finally the specific weight is equal to: $\gamma = \SI{9770}{\N/\m^3}$.

To obtain the density $\rho$ we use $\gamma$. These are related by the
gravitational constant $g$ (Specific weight is a measure of the force
exerted on a unit volume of fluid by gravity).
\begin{align*}
  \gamma &= \rho\,g \\
  \therefore \rho &= \frac{\gamma}{g} =
                    \frac{\SI{9770}{\N\per\m^3}}{\SI{9.81}{\m\per\s^2}}
                    = \SI{996}{\N\cdot{}\s^2\per\m^4} = \SI{996}{\kg/\m^3}
\end{align*}

We now calculate the specific gravity using the density of water at
\SI{4}{\celsius} as a reference density:

\begin{equation*}
  SG = \frac{\rho}{\rho_{H_2O@\SI{4}{\celsius}}} = \num{0.996}
\end{equation*}
\end{solution}

\begin{question}
  The temperature and pressure at the surface of Mars during a Martian
  spring day were determined to be \SI{-50}{\celsius} and \SI{900}{\Pa},
  respectively.
  \begin{enumerate}[label=\alph*)]
  \item Determine the density of the Martian atmosphere for these
    conditions if the gas constant for the Martian atmosphere is
    assumed top be equivalent to that of carbon dioxide.
  \item Compare the answer from part ($a$) with the density of the
    earth's atmosphere during a spring day when the temperature is
    \SI{18}{\celsius} and the pressure \SI{101.6}{\kPa} (abs).
  \end{enumerate}
\end{question}

\begin{question}
  Calculate the Reynolds numbers for the flow of water and for air
  through a \SI{4}{\mm} diameter tube, if the mean velocity is
  \SI{3}{\m/\s} and the temperature is \SI{30}{\celsius} in both
  cases. Assume the air is at standard atmospheric pressure.
\end{question}

\begin{question}
Estimate the increase in pressure (in psi) required to decrease a unit
volume of mercury by \SI{0.1}{\percent}
\end{question}

\begin{question}
Often the assumption is made that the flow of a certain fluid can be
considered as incompressible flow if the density of the fluid changes
by less than \SI{2}{\percent}. If air is flowing through a tube such
that the air pressure at one section is \US{9.0}{\psi} (gauge) and at a
downstream section it is \US{8.5}{\psi} (gauge) at the same
temperature, do you think that this flow could be considered an
incompressible flow? Support your answer with the necessary
calculations. Assume standard atmospheric pressure.
\end{question}

\begin{question}
  Natural gas at \SI{70}{\degree{F}} and standard atmospheric pressure
  of \SI{14.7}{\psi} (abs) is compressed isentropically to a new
  absolute pressure of \SI{70}{\psi}. Determine the final density and
  temperature of the gas.
\end{question}

\begin{question}
  Compare the isentropic bulk modulus of air at \SI{101}{\kPa} (abs)
  with that of water at the same pressure.
\end{question}
